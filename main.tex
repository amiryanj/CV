%\documentclass{article}
\documentclass[10pt]{res}

\usepackage{helvet}
\usepackage{hyperref}
\usepackage{bibentry}
\usepackage[utf8]{inputenc}
\usepackage[english]{babel}

\title{Javad Amirian}
%\subtitle{Resume}
\author{Javad Amirian}
\date{2016-05-30}

\newsectionwidth{0pt}

\begin{document}

\name{Javad Amirian\\ \\}
\vspace{28pt}

%\section{\centerline{Contact Information}}

%\address{{\bf Permanent Address} \\ \\  \\  % Your address 1
\address{{\bf Address} \\ No. 210, 1 Rue Jean Guy\\ Rennes, France\\
	Postal Code: 35000} % Your address 1


\address{{\bf Contact} \\ 
Mobile: +33 6 32 38 50 50\\
mail: amiryanj@ce.sharif.edu \\
mail: javad.amirian@inria.fr \\
http://\url{team.inria.fr/rainbow/javad-amirian/} } % Your address 2


\begin{resume}

% Objective Section
\section{\centerline{Summary}}
\vspace{8pt} % Gap between title and text
I'm a Robotics and AI enthusiast that dedicated 7 years of my life to design, build and develop intelligent mobile robots. Having successfully completed formal studies in Electronics and then graduated from Sharif University of technology in AI, I'm currently PhD student at Inria in Brittany, France working on European robotics project called CrowdBot.


%To make contribution in the field of soccer robots to fulfill dream of Robocup 2050. I have  And the main ground for changing my major in M.Sc. was to acquire knowledge in different fields related to robotics.



\section{\centerline{Research Interests}}
\vspace{8pt} % Gap between title and text

\begin{itemize} \itemsep -1pt
\item Robot and Machine Vision
\item Motion Planning and Mobile Robot Navigation
\item Multi-agent Systems
\item Machine Learning and Graphical Models
\end{itemize}

\section{\centerline{Education}}
\vspace{8pt} % Gap between title and text
{\sl PhD Student}\\
University of Rennes 1, Brittany, France \hfill 2018-2020\\
Title: Short-term Pedestrian Prediction for Mobile Robots in Crowded Space


{\sl Master of Science} in Computer Engineering (Artificial Intelligence) \\ 
Sharif University of Technology, Tehran, Iran \hfill 2012 - 2014\\ 
%THESIS - Job Evaluation: Dynamic Motion Planning and Obstacle Avoidance Simulation for Autonomous Robot-Car in Webots \\
GPA: 16.66/20\\
Ranked 11th in nationwide M.Sc. university entrance exam for Artificial Intelligence, Iran \hfill 2012
 
{\sl Bachelor of Science} in Electrical Engineering (Electronics) \\ 
Shahid Behesti University (National University of Iran), Tehran, Iran \hfill 2007 - 2012\\
GPA: 15.44/20\\
Top 0.6\% nationwide entrance exam of Iranian Universities, among more than 300,000 participants. \hfill 2007

{\sl Diploma} in Mathematics and Physics \\
Tiz-Hooshan School (National Org for Development of Exceptional Talents), Kermanshah, Iran \hfill 2000 - 2007\\
GPA: 18.00/20


%-----------------------------------------------------------
\vspace{-8pt} % Negative space between title and text
 

\section{\centerline{Graduate Courses}}
\vspace{8pt} % Gap between title and text
Machine Learning, Statistical Pattern Recognition, Computational Intelligence, Image Processing, Planning in Artificial Intelligence, Speech Processing, Stochastic Processes.


% Main Projects
\section{\centerline{Professional Experience}}
\vspace{-6pt} % Negative space between title and text

% describe small size project
\paragraph{Team Leader @ \href{http://robocup.sbu.ac.ir}{Cyrus Robotics}}(Small-Size Soccer Team of Shahid Beheshti Univ.) \hfill 2010 - 2015\\
%A team of 6 small mobile robots play soccer on a green carpeted field that is 9 m long by 6 m wide. Small Size robot soccer focuses on the problem of intelligent multi-agent cooperation and control in a highly dynamic environment with a hybrid centralized/distributed system. 
Besides experiencing a professional team work management of seven graduate and undergraduate students, I heve been designing and developing electronic boards and embedded systems of the robots as electronic designer until 2012. During this period, a small wireless network was formed using Zigbee technology to enable communication between robots. Decision commands sent from Off-field computer were parsed and executed with ARM-based micro-controller that run a PI controller to  drive the robot wheels at a desired speed and trigger the kicker module.\\
After 2012 I joined the software group and rewritten AI codes to c++. I developed the motion planning module based on an idea in my MSc thesis. We also introduced an approach to improve the robot movements and compensating inaccuracies in robot building using a Takegi-Sugeno fuzzy method.
In 2014 we implemented a speech recognition system to convert human voice signals into robot commands in order to facilitate robot learning.

\paragraph{Computer Vision Expert @ Sepehr Etelaat Co.} (Related to Sharif Univ.) \hfill May 2015 - present\\
Sepehr is a computer vision based company that develop solutions in sport analysis. My job is developing image processing and vision algorithms and providing high-level API for UI programmers. My current project is a software for adding virtual advertisement to sport videos. A main processing step in this system is to estimate camera calibration based on corresponding points in soccer pitch. The vision engine has several processing modules including camera position estimation, camera pose exhaustive search for initializing pose tracker, shot boundary detection, Levenberg–Marquardt Based Optimizer and etc. Many AI techniques are utilized in this software such as GA, Kalman Smoothing, Optical Flow etc.

\paragraph{Computer Vision Expert @ \href{http://www.spadsystem.com/}{Spad System Co.} } \hfill April 2016 - present\\
Spadsystem is a startup software company that was founded in 2013. It specializes in accounting and management tools. Company started developing an automatic license plate recognition system for Iranian vehicles \href{http://didbaan.com}{(Didbaan)} and I joined the team in 2016 as a vision expert. My role is developing vision algorithms and also revising existing codes with OpenCV. I designed a tool for detecting moving objects using motion segmentation technique and also optimized an existing LBP descriptor to find the plates regions in the image. Furthermore I implemented a Persian OCR module using standard C++ libraries to extract the content of the plates.


%\section{\centerline{Work Experience}}
\section{\centerline{Teaching Experience}}
\vspace{15pt}
\begin{itemize} \itemsep -1pt

\item {\sl Teaching Assistant} of Numerical Optimization course \hfill Spring 2017\\
Dr. S. Hamid Amiri (Shahid Rajaee University) 

\item {\sl Teaching Assistant} of Machine Vision course (CE-40687) \hfill Spring 2014\\
Dr. Mansour Jamzad (Sharif University of Technology)

\item {\sl Teaching Assistant} of Digital Design course \hfill Spring 2010, Spring 2011\\
Dr. Somayeh Timarchi (Shahid Beheshti University)

\item {\sl Lecturer} of Introduction to AVR Microcontrollers \hfill 2009-2011\\
At Scientific Association of Electrical Engineering (Shahid Behshti University) 

\end{itemize}



\section{\centerline{Publications}}
\vspace{15pt} % Gap between title and text
%{\bf Conference Papers}\\

\begin{itemize} \itemsep -1pt % Reduce space between items

\item (TDP) {\bf J. Amiryan,} S. Raeessi, P. Payandeh, B. Nadimi, N. Nouri, M. R. Kamali, E. Nazemi, {\bf ''CYRUS 2016 Team Description Paper,''} 2016.

\item {\bf J. Amiryan,} M. Jamzad, {\bf ''Adaptive motion planning with artificial potential fields using a prior path,''} Robotics and Mechatronics (ICROM), 2015 3rd RSI International Conference on, 2015.

\item (MSc Thesis) {\bf J. Amiryan,} {\bf ''Dynamic Motion Planning and Obstacle Avoidance Simulation for Autonomous Robot-car in Webots,''} MSc Thesis, Department of Computer Engineering, Sharif University of Technology, August 2014.

\item M. Nazari, {\bf J. Amiryan,} E. Nazemi, {\bf ''Improvement of robot navigation using fuzzy method,''} AI and Robotics and 5th RoboCup Iran Open International Symposium (RIOS), 2013 3rd Joint Conference of. IEEE, 2013.

\item J. Mazloum, A. Jalali, {\bf J. Amiryan,} {\bf ''A novel bidirectional neural network for face recognition,''} Computer and Knowledge Engineering (ICCKE), 2012 2nd International eConference on. IEEE, 2012.

\item (BSc Project Report) {\bf J. Amiryan,} J. Kamali, {\bf ''Automatic Traffic Control System With Police Robot,''} BSc.
Thesis, Department of Electrical Engineering, Shahid Beheshti University, March 2012.
\end{itemize}

\section{\centerline{Skills}}
\vspace{15pt}
\begin{itemize} \itemsep -1pt

\item {\bf OS:} Microsoft Windows; Linux (Ubuntu, Fedora, CentOS); Android.
\item {\bf Office Tools:} \LaTeX; Libre Office; Microsoft Word, Excel, PowerPoint, Visio.
\item {\bf Graphic Tools:} Adobe Photoshop, Adobe Premiere; 3ds Max.
\item {\bf Software Design:} OOD; Concurrent and Multithread; Modular Programming.
\item {\bf Programming Languages:} C/C++, Matlab, Python, Java.
\item {\bf Database:} PostgreSQL, ODB (ORM).
\item {\bf Version Control:} git.
\item {\bf AI and Robotics Tools:} OpenCV, Image Processing Toolbox, Webots Simulator, Weka.
\item {\bf Embedded Design:} ARM Processors (AT91, LPC), Atmel AVR Family, C51 Family; Altium Designer, Proteus, Protel.

\end{itemize}



\section{\centerline{Activities}}
\begin{itemize}
\item I love robots and robotic competitions, I have participated in many robotic events in fields of intelligent mouses and soccer robots.
\item I am a member of technical committee (TC) of Small Size League at Robocup Iran Open. \href{http://2017.iranopen.ir/en-gb/leagues/robocupsoccer/smallsizerobotleague.aspx}{Link}.

\item I am strongly interested in Football and am a fan of Perspolis club. Going to stadium and encouraging my favorite team is one of my most exciting entertainments.

\item I do Body-building, Mountaineering, play futsal, and also some computer sport games such as FIFA, PES and NFS.

\end{itemize}


\section{\centerline{References}}

\begin{itemize}
\item   {\bf \href{http://people.rennes.inria.fr/Julien.Pettre} {Dr. Julien Pettre} }\\
Research Scientist at Rainbow, INRIA-Rennes, Brittany, France\\
Email: julien.pettre@inria.fr

\item   {\bf \href{http://ce.sharif.edu/~jamzad} {Dr. Mansour Jamzad} }\\
Department of Computer Engineering, Sharif University of Technology,Tehran, Iran\\
Email: jamzad@sharif.edu

\item   {\bf  Dr. Eslam Nazemi}\\
Department of Electrical and Computer Engineering, Shahid Beheshti University, Tehran , Iran\\
Email: nazemi@sbu.ac.ir

\item   {\bf \href{http://facultymembers.sbu.ac.ir/eshghi/}{Dr. Mohamad Eshghi} }\\
Department of Electrical and Computer Engineering, Shahid Beheshti University, Tehran , Iran\\
Email: m-eshghi@sbu.ac.ir

\end{itemize}

\end{resume}


\end{document}
