%\documentclass{article}
\documentclass[10pt]{res}

\usepackage{helvet}
\usepackage{hyperref}
\usepackage{bibentry}
\usepackage{csquotes}
\usepackage[utf8]{inputenc}
\usepackage[english]{babel}

\title{Javad Amirian}
%\subtitle{Resume}
\author{Javad Amirian}
\date{2016-05-30}

\newsectionwidth{0pt}

\begin{document}

\name{Javad Amirian\\ \\}
\vspace{28pt}

%\section{\centerline{Contact Information}}

%\address{{\bf Permanent Address} \\ \\  \\  % Your address 1
\address{{\bf Address} \\ 1 Rue Maurice Arnoux \\
	 92120 Montrouge, France } % Your address 1


\address{{\bf Contact} \\ 
Mobile: +33 6 32 38 50 50\\
%mail: amiryanj@ce.sharif.edu \\
Email: amiryan.j@gmail.com \\
%\url{https://linkedin.com/in/javad-amirian-68a3a584/} \\ % Your address 2
%\url{https://github.com/amiryanj} 
} % Your address 2

\begin{resume}


% \section{\centerline{Research Interests}}
% \vspace{8pt} % Gap between title and text

% \begin{itemize} \itemsep -1pt
% \item Robot Vision
% \item Deep Learning and Generative Models
% \item Crowd Simulation and Human Motion Prediction
% \end{itemize}

\hrulefill

\section{\centerline{Education}}
\vspace{8pt} % Gap between title and text

$\bullet$ \textbf{{PhD} in Computer Science (Robotics and AI)} \hfill Jan. 2018 - Jul 2021\\
Inria (Rainbow Team), Rennes, France \\
Thesis: Human Motion Trajectory Prediction for Robot Navigation (CrowdBot)\\
Supervisors: Dr. Julien Pettr\'e , Dr. Jean-Bernard Hayet \\
\hspace{-14pt}
%Jury: Dr. Alexandre Alahi (EPFL), Dr. Dinesh Manocha (University of Maryland), Dr. Nuria Pelechano (Polytechnic University of Catalonia), Dr. Frédéric Lerasle (Toulouse III), Dr. Eric Marchand (University of Rennes)

% ===================== Master =============================
$\bullet$ \textbf{{MSc} in Computer Engineering (Artificial Intelligence)} \hfill Sept. 2012 - Sept. 2014\\ 
Sharif University of Technology, Tehran, Iran (1st Rank Technical University in Iran)\\
Thesis: Dynamic Motion Planning and Obstacle Avoidance for Simulated Autonomous Car in Webots \\
Supervisor: Dr. Mansour Jamzad\\
%GPA: 16.66/20\\
\vspace{-14pt}
%\textit{Graduate Courses}: Machine Learning, Statistical Pattern Recognition, Computational Intelligence, Image Processing, Planning in Artificial Intelligence, Speech Processing, Stochastic Processes.\\

%\vspace{-18pt}
%* Ranked \underline{11th} in nationwide M.Sc. university entrance exam for Artificial Intelligence, Iran \hfill 2012


% =================== Bachelor ============================= 
$\bullet$ \textbf{{BSc} in Electrical Engineering (Electronics)} \hfill Sept. 2007 - Sept. 2012\\ 
Shahid Behesti University (National University of Iran), Tehran, Iran  \\
%GPA: 15.44/20\\

%\vspace{-18pt}
%* Top 0.6\% nationwide entrance exam of Iranian Universities among $\sim$300,000 participants. \hfill 2007

%{\sl Diploma} in Mathematics and Physics \\
%Tiz-Hooshan School (National Org for Development of Exceptional Talents), Kermanshah, Iran \hfill 2000 - 2007\\
%GPA: 18.00/20


%-----------------------------------------------------------
\vspace{-12pt} % Negative space between title and text
 

%\section{\centerline{Graduate Courses}}
%\vspace{8pt} % Gap between title and text
%Machine Learning, Statistical Pattern Recognition, Computational Intelligence, Image Processing, Planning in Artificial Intelligence, Speech Processing, Stochastic Processes.

\hrulefill

% Main Projects
\section{\centerline{Professional Experience}}
\vspace{-12pt} % Negative space between title and text

\paragraph{$\bullet$ Postdoctoral Researcher @ \href{https://isir.upmc.fr/} {ISIR (Robotics Lab - Sorbonne University)}} \hfill Sep. 2023 - Oct. 2024\\
As part of the euROBIN project, I focused on developing social navigation algorithms for mobile robots such as Pepper and Tiago. This included integrating Vision-Language Models (VLMs) and LLMs into the robot navigation stack using ROS, in a collaboration project with the LAAS robotic center, to advance the robots capabilities for human-aware navigation.
\vspace{-12pt}

\paragraph{$\bullet$ CTO and Head of AI @ \href{http://vivetennis.com} {Vive Robotics}} \hfill Mar. 2021 - Aug 2023\\
As the CTO and Head of AI, I led the development of the Vive Tennis, a tennis-ball-retriever robot that utilizes advanced Computer Vision and AI algorithms to efficiently navigate on the court, locate tennis balls, and identify players. My primary responsibility was to oversee the delivery of the MVP, by setting specific objectives for each principle. I guided the AI team in developing a high-fps video processing pipeline on-the-edge that maximizes the robot's agility while ensuring cost-effective hardware solutions to maintain an affordable product price.
\vspace{-12pt}

\paragraph{$\bullet$ Co-Founder of \href{http://decorar.ai} {DecorAR} @ Inria Startup Studio} \hfill April. 2022 - Mar. 2023\\
DecorAR is a tech project incubated at Inria Startup Studio to bring AI and Visual Recommendation to Augmented Reality environments. "\textit{Interior Design by AI}" is the first platform being developed by DecorAR and it addresses the problem of finding compatible pieces of furniture among a vast database of products from different categories and brands. As a CTO, my job was to build ML models and develop the stack to make the PoCs accessible to early testers.
\vspace{-12pt}

\paragraph{$\bullet$ Doctoral Researcher @ \href{http://inria.fr} {Inria}} (EU H2020 Project) \hfill Jan. 2018 - Jul. 2021\\
I contributed to developing tools for \textit{Motion Prediction} of pedestrians around robots to enhance navigation in crowded environments. Leveraging GANs, I developed the models to predict multi-modal distributions of human trajectories. This work honed my expertise in ML frameworks like PyTorch and Keras, and large-scale data analysis, with significant focus on neural network training and visualization.
\vspace{-8pt}

\paragraph{$\bullet$ Computer Vision Engineer @ \href{http://pixballsports.com}{PixBall}} (Previously Sepehr) \hfill May 2015 - Dec. 2017\\
At PixBall, I contributed in developing high-performance camera calibration and optimization algorithms, mostly in C++ and Matlab, as a computer vision engineer. 
%  accuracy in overlay placement and created systems to handle variable lens distortions at different zoom levels.
I developed novel algorithms to handle camera lens distortion and this contributed significantly to improving the accuracy of embedding graphical overlays and virtual advertisements in sport content.
\vspace{-12pt}

% \paragraph{$\bullet$ Computer Vision Engineer @ \href{http://www.spadsystem.com/}{Spad System Co.} } \hfill Apr. 2016 - Nov. 2017\\
% At Spad System Co., I played a key role in the \href{http://didbaan.com}{(Didbaan)} project, focusing on automatic recognition of Iranian license plates using the OpenALPR engine based on OpenCV. I utilized my expertise in computer vision and image processing to optimize and fine-tune the code for accurate detection and recognition of local license plates. Additionally, I spearheaded the data collection effort, curating a dataset of over 5,000 annotated license plate images. This highly successful implementation of Didbaan at multiple parking sites revolutionized vehicle entrance and exit management, underscoring my proficiency in computer vision and image processing.



% \paragraph{$\bullet$ Founder and Team Leader @ {Cyrus}} (Students Robotics Team) \hfill Jul. 2010 - Apr. 2015\\
% During my undergraduate studies, I co-founded \textbf{Cyrus}, an ambitious project centered around small-size soccer robots. I assembled and led a dedicated team of 5-10 students from diverse backgrounds in electrical engineering, computer engineering, and mechanical engineering. Over the course of 5 years, I guided the project's direction by conducting thorough research, designing the system architecture, and on-boarding talented students to contribute to its development. Initially, I took charge of electronic design, robot firmware, and in 2012, I transitioned to focus on AI and robot navigation and ultimately creating motion planning and control algorithms for the robots. As a notable research outcome, our team introduced a novel approach utilizing Fuzzy logic to enhance robot movements and compensate for inaccuracies in the robot hardware. Additionally, I proposed an efficient motion-planning algorithm that optimizes multiple objectives simultaneously, allowing for adaptive navigation parameters.



%\section{\centerline{Work Experience}}
%\section{\centerline{Teaching Experience}}
%\vspace{15pt}
%\begin{itemize} \itemsep 1pt

%\item {\sl Teaching Assistant} of Numerical Optimization course \hfill Spring 2017\\
%Dr. S. Hamid Amiri (Shahid Rajaee University) 

%\item {\sl Teaching Assistant} of Machine Vision course (CE-40687) \hfill Spring 2014\\
%Dr. Mansour Jamzad (Sharif University of Technology)

%\item {\sl Teaching Assistant} of Digital Design course \hfill Spring 2010, Spring 2011\\
%Dr. Somayeh Timarchi (Shahid Beheshti University)

%\item {\sl Lecturer} of Introduction to AVR Microcontrollers \hfill 2009-2011\\
%At Scientific Association of Electrical Engineering (Shahid Behshti University) 

%\end{itemize}

\hrulefill

\section{\centerline{List of Publications}}
\vspace{15pt} % Gap between title and text
%{\bf Conference Papers}\\

\begin{itemize} \itemsep 2pt % Reduce space between items


\item Gheisari, M., {\bf Amirian, J.}, Furon, T. Amsaleg, L., {\bf ''AggNet: Learning to Aggregate Faces for Group Membership Verification,''} Signal Processing: Image Communication, 2025.

\item {\bf Amirian, J.}, Abrini, M., Chetouani, M., {\bf ''Legibot: Generating Legible Motions for Service Robots Using Cost-Based Local Planners,''} International Conference on Robot and Human Interactive Communication (\textbf{ROMAN-2024}), Aug. 2024.

\item Magri, P., {\bf Amirian, J.}, Chetouani, M., {\bf ''Upgrading Pepper Robot Social Interaction with Advanced Hardware and Perception Enhancements,''} 16th International Conference on Social Robotics (\textbf{ICSR-2024}), Aug. 2024.

\item Zhang, B., \textbf{Amirian, J.} Eberle, H., Pettr\'e, J., Holloway, C., Carlson , T. \textbf{"Towards Safe Human-Robot Interactions in Crowds: Empirical Study of Pedestrian Dynamics with a Wheelchair and a Pepper Robot."} International Journal of Social Robotics (\textbf{SORO-2022}).
	
\item {\bf Amirian, J.}, Hayet, J. B., Pettré, J., {\bf ''What we see and What we don’t see: Imputing Occluded Crowd Structures from Robot Sensing,''} (\textbf{Preprint-2021}).

\item {\bf Amirian, J.}, Zhang, B., Valente Castro, F., Baldelomar, J., Hayet, J. B., Pettré, J. {\bf ''OpenTraj: Assessing Prediction Complexity in Human Trajectories Datasets.''} In Proceedings of the 15th Asian Conference on Computer Vision (\textbf{ACCV-2020}), Nov-Dec. 2020.

\item van Toll, W., Grzeskowiak, F., Gandía, A.L., \textbf{Amirian, J.}, Berton, F., Bruneau, J., Daniel, B.C., Jovane, A. and Pettré, J., ''\textbf{Generalized Microscropic Crowd Simulation using Costs in Velocity Space},'', In Symposium on Interactive 3D Graphics and Games (\textbf{I3D-2020}), May 2020.

\item {\bf Amirian, J.}, Van Toll, W., Hayet, J. B., Pettré, J. {\bf ''Data-Driven Crowd Simulation with Generative Adversarial Networks.''} In Proceedings of the 32nd International Conference on Computer Animation and Social Agents (\textbf{CASA'19}), Jul. 2019.

\item {\bf Amirian, J.}, Hayet, J. B., Pettré, J., {\bf ''Social ways: Learning multi-modal distributions of pedestrian trajectories with GANs,''} IEEE Conference on Computer Vision and Pattern Recognition (\textbf{CVPR-2019}) Precognition Workshop, Jul. 2019.

%\item (TDP) {\bf Amiryan, J.}, Raeessi, S., Payandeh, P., Nadimi, B., Nouri, N., Kamali, M. R., Nazemi, E., {\bf ''CYRUS 2016 Team Description Paper,''} 2016.

\item {\bf Amiryan, J.}, Jamzad, M., {\bf ''Adaptive motion planning with artificial potential fields using a prior path,''} 3rd RSI International Conference on Robotics and Mechatronics (\textbf{ICROM}), 2015.

%\item (MSc Thesis) {\bf Amiryan, J.}, {\bf ''Dynamic Motion Planning and Obstacle Avoidance Simulation for Autonomous Robot-car in Webots,''} \textbf{MSc Thesis}, Department of Computer Engineering, Sharif University of Technology, August 2014.

\item Mazloum, J., Jalali, A., {\bf Amiryan, J.}, {\bf ''A novel bidirectional neural network for face recognition,''} 2nd International eConference on Computer and Knowledge Engineering (\textbf{ICCKE}), 2012.

%\item (BSc Project Report) {\bf J. Amiryan,} J. Kamali, {\bf ''Automatic Traffic Control System With Police Robot,''} BSc. Thesis, Department of Electrical Engineering, Shahid Beheshti University, March 2012.
\end{itemize}


%\section{\centerline{Professional Services}}
%\vspace{15pt}
%
%\begin{itemize} \itemsep -1pt
%%	\item [] \textbf{Journal and Conference Reviews:}
%	\item \textbf{PC Member} of \textit{ICRA Workshop on
%		Long-term Human Motion Prediction} / 2021
%	\item \textbf{Reviewer} @ Computer Animation and Virtual Worlds / 2021
%	\item \textbf{Reviewer} @ \textit{IROS} [4 papers] / 2021
%	\item \textbf{Reviewer} @ \textit{IEEE Robotics and Automation Letters (RA-L)} / 2020
%	\item \textbf{Reviewer} @ \textit{Computer Vision and Image Understanding} / 2020
%
%	\item \textbf{Reviewer} @ \textit{IEEE Transactions on Neural Networks and Learning Systems} / 2020
%
%	\item \textbf{Subreviewer} @ \textit{SIGGRAPH} / 2020
%
%	\item \textbf{Subreviewer} @ \textit{CASA} (Conference on Computer Animation and Social Agents) / 2018-2019
%
%	\item \textbf{PC Member} of CrowdNav (IROS Workshop on Robot Navigation in Crowd) / 2018
%
%%	\item \textbf{Subreviewer} @ \textit{CASA'18} (Conference on Computer Animation and Social Agents)
%
%	\item \textbf{Technical Committee} @ Robocup Iran Open - Small Size League / 2014 - 2017
%
%\end{itemize}

\hrulefill

\section{\centerline{Technical Skills}}
\vspace{15pt}
%\begin{itemize} \itemsep -1pt

%%%\item {\bf OS:} Microsoft Windows; Linux (Ubuntu, Fedora, CentOS); Android.
%%%\item {\bf Office Tools:} \LaTeX; Libre Office; Microsoft Word, Excel, PowerPoint, Visio.
%\item {\bf Programming Languages:} Python, C/C++, Qt, Matlab, Swift.
%\item {\bf Deep Learning:} Pytorch, Keras (TensorFlow) | RNNs, GANs, CNNs.
%\item {\bf Version Control:} Git, GitHub.
%\item {\bf AI and Robotics Tools:} OpenCV, Nvidia Jetson, ROS, CARLA, Webots, Unity, Gazebo.
%\item {\bf Software Development:} OOD; Concurrent and Multithread; Modular Programming.
%\item {\bf Project Management:} ClickUp, Trello.
%\item {\bf Database:} PostgreSQL, ODB (ORM).
%\item {\bf Graphic Tools:} Adobe Photoshop, Adobe Premiere; 3ds Max.
%%%\item {\bf Embedded Design:} ARM Processors (AT91, LPC), Atmel AVR Family, C51 Family; Altium Designer, Proteus, Protel.

%\end{itemize}


\begin{itemize} \itemsep 1pt

\item {\bf Software Development:} Proficient in \textbf{Python} and \textbf{C/C++}, employing best practices in software development. Skilled in architecting, implementing and debugging robust and scalable code using \textbf{object-oriented design (OOD)} and modular approaches.

\item {\bf Version Control \& Project Management:} Extensive experience with \textbf{Git} and \textbf{GitHub}, utilizing version control best practice principles and workflows to streamline development processes and leveraging \textbf{CI/CD} pipelines for automated testing and deployment.

\item {\bf Deep Learning:} Proficient in \textbf{PyTorch} / \textbf{Keras} and experience designing and optimising NN models. Leveraging ML tools like \textbf{W\&B} and \textbf{FiftyOne} for data/model management, versioning, and experiment tracking. Experience in visualizing model performance, analyzing data distributions, and streamlining machine learning workflows.

\item {\bf Computer Vision \& 3D Perception:} Proficient in leveraging \textbf{OpenCV} and ML solutions for extracting insights from visual data and perform \textbf{Object Detection}, \textbf{segmentation} and \textbf{tracking}. Experience in multiple \textbf{Camera Calibration} problems for \textbf{3D reconstruction} and \textbf{Localization}.

\item {\bf Robotics:} Proficient in working with \textbf{ROS}, deep experience with \textbf{Nvidia Jetson} family and its frameworks (DeepStream, TensorRT), and simulation environments like \textbf{Carla}, \textbf{Gazebo}, \textbf{Unity}, and \textbf{Webots} to prototype intelligent and autonomous robotic systems.

\item {\bf Cloud \& Containerization:} Experience with \textbf{AWS} tools for scalable and reliable cloud solutions and \textbf{Dockers} and \textbf{Containerization} for efficient deployment. Extensive experience in real-time image processing pipelines on cloud, optimizing performance and ensuring reliable data processing.

\item {\bf Linux \& Bash:} Expertise in Linux and \textbf{Bash scripting} for automation and system management. Skilled in developing and optimizing scripts and services for various tasks.

\item {\bf Web \& App Development:} Skilled in \textbf{Django} web framework. Proficient in modeling \textbf{SQL databases}, UI design, and template development. Additionally, experienced in iOS app development using \textbf{Swift} for engaging user experiences and \textbf{Figma} for prototyping.

\end{itemize}


% \section{\centerline{Hobbies and Other Activities}}
% \begin{itemize}
% %\item I love robots and robotic competitions, I have participated in many robotic events in fields of intelligent mouses and soccer robots.
% \item Occasional Blogging about AI, Tech, and Society (Instagram, Telegram)
% \item Hobby photography and videography
% \item Interested in football (soccer), hiking (if there is some mountain around), cycling, and power-lifting, and also supporting my favorite club: F.C. Perspolis club (Iran). 

% \end{itemize}

\hrulefill

\section{\centerline{References}}


\begin{itemize}
% \item   {\bf \href{http://people.rennes.inria.fr/Julien.Pettre} {Dr. Mohammad Rouhani} }\\
% CEO at DecorAR, Paris, France\\
% Email: mohammad.rouhani@inria.fr

\item   {\bf \href{https://www.isir.upmc.fr/personnel/chetouani/} {Dr. Mohamed Chetouani} }\\
Full Professor and Deputy Director at ISIR, Sorbonne University, Paris, France\\
Email: mohamed.chetouani@isir.upmc.f

\item   {\bf \href{http://people.rennes.inria.fr/Julien.Pettre} {Dr. Julien Pettre} }\\
Research Scientist at Rainbow, Inria, Rennes, France\\
Email: julien.pettre@inria.fr


\item   {\bf \href{http://aplicaciones.cimat.mx/Personal/jbhayet} {Dr. Jean-Bernard Hayet} }\\
Researcher at CIMAT, Department of Computer Science., Guanajuato, Mexico\\
Email: jbhayet@cimat.mx


% \item   {\bf \href{http://ce.sharif.edu/~jamzad} {Dr. Mansour Jamzad} }\\
% Department of Computer Engineering, Sharif University of Technology, Tehran, Iran\\
% Email: jamzad@sharif.edu


%\item   {\bf  Dr. Eslam Nazemi}\\
%Department of Electrical and Computer Engineering, Shahid Beheshti University, Tehran , Iran\\
%Email: nazemi@sbu.ac.ir

%\item   {\bf \href{http://facultymembers.sbu.ac.ir/eshghi/}{Dr. Mohamad Eshghi} }\\
%Department of Electrical and Computer Engineering, Shahid Beheshti University, Tehran , Iran\\
%Email: m-eshghi@sbu.ac.ir

\end{itemize}

\end{resume}


\end{document}
